\documentclass[11pt]{article}
\usepackage[utf8]{inputenc}
\usepackage{fullpage}
\usepackage[colorlinks,citecolor=blue,urlcolor=blue,bookmarks=false,hypertexnames=true]{hyperref}
\title{Some Problems in Curvature Flows of Hypersurfaces}
\author{Devesh Rajpal}
\date{}

\begin{document}
\maketitle
\begin{center}
   \textbf{Thesis Proposal for Mathematics PhD} 
\end{center}


The Mean curvature Flow (MCF) evolves hypersurfaces immersed in Euclidean space in their normal direction with a speed equal to the mean curvature at each point. It bends the higher curved parts with more speed than the lower curved parts in order to uniformize the curvature across the hypersurface. Also, the parabolic nature of the equation gives short time existence and uniqueness; so given a hypersurface we have one way to evolve to possibly study its geometry. %It is also the negative gradient flow of the area functional, so compact hypersurfaces flow in the direction of the steepest descent for the area and become extinct in finite time. 
However, compact hypersurfaces encounter a singularity as they become non-smooth in places in finite time (where curvature blows up).
A lot of progress has been made to understand this flow and its nature of singularities.

Besides, there has been considerable interest in flow by curvature functions of hypersurfaces. Mean curvature flow is a linear example of it, but there are others as well such as Gauss curvature flow, inverse mean curvature flow, and general fully non-linear flow. A crucial aspect in the study of such flows is the search for monotonous quantities and singularity analysis. 

The goal of my PhD is to work on the following problems in mean curvature flow and other non-linear flows - 

\begin{enumerate}
    \item The Stampacchia iteration has been used successfully in mean curvature flow and other non-linear flows in a number of places in order to utilize a good diffusion term to get bounds. An alternate approach to this is to consider the second fundamental form $ h_{ij} $ as a scalar function of the unit tangent bundle $ STM $ defined by $ f(x,v,t) = h_{ij}(x,t)v^{i}v^{j} $. %It can be shown that $ f $ satisfies a degenerate parabolic equation on $ STM $ and that the operator is sub-elliptic at points where the curvature is not zero. 
    It will be interesting to look for `Harnack estimates' on solutions of parabolic equations of this kind and try to deduce results such as convergence of convex surfaces to spheres or to explore asymptotic convexity for mean-convex surfaces. %It would be interesting to look for applications of maximum principle in this setting.
    Also, it may be helpful to understand the behavior of self-similar solutions in this setting in order to study the general singularities.  
    \item The above strategy would also allow us to understand the behavior of hypersurfaces (particularly convex hypersurfaces) evolving by speeds that are not smooth, such as $F=\kappa_{\max}+cH$.  It is expected that convex hypersurfaces should become round under these flows, but the non-smoothness of the speed prevents the application of the Stampacchia method which uses integration by parts.  The proposed method would get around this difficulty.
    \item Beyond that, it would be interesting to explore motion by the maximum principal curvature.  This flow is not only non-smooth, but also badly degenerate, so existence and regularity are difficult questions, but the approach outlined might shed some light on this.
\end{enumerate}


\bibliographystyle{plain} % We choose the "amsalpha" reference style
\bibliography{proposal2}
\nocite{*}

\end{document}