\chapter{Preface}

\section*{Why Mean curvature flow}

%This question has been bugging me for past few months.
Why do we want to study this flow and if we do what results can we achieve? To start with it is a very natural flow to consider on hypersurfaces in Euclidean space. It bends the higher curved parts with more speed than lower curved parts in order to uniformize the curvature across the hypersurface. Also, the parabolic nature of the equation directly gives short time existence and uniqueness; so we know given a hypersurface we have one way to evolve to possibly study its geometry. For its twin ``Ricci flow'' as Huisken calls it the motivation was uniformizing Riemannian manifolds with an eye towards Poincar\'e conjecture. This is of-course with benefit of hindsight after Perelman's seminal resolution using surgery methods. 

%So the question begs itself : 
What do we want to do with the Mean curvature flow on hypersurfaces? A generic answer is to study the geometry of hypersurfaces and attempt a classification. This is severely restricted by the assumption of mean convexity ($ H>0 $ everywhere) which makes maximum principle work in a number of cases. Huisken's result on the convergence of convex hypersurfaces into round sphere is the first step towards it, but it doesn't achieve much topologically. A uniformly convex hypersurface is diffeomorphic to unit sphere by Gauss map to begin with. For non-convex hypersurface singularities might develop which prohibit a direct analysis. To overcome this we blow-up the manifold near singularity and this limiting process gives an ancient solution. So we shift our attention to a classification of ancient solutions which is still a difficult problem. Angenent-Daskalopoulos-Sesum and Brendle-Choi have obtained results in this direction without self-similarity conditions

Another direction the Mean curvature flow is being explored is the Lagrangian Mean curvature flow in order to find special Lagrangians inside symplectic manifolds. In the case of Calabi-Yau manifolds the condition of being Lagrangian is preserved under Mean curvature flow.

Mean curvature flow can also be potentially used to find nice codimension one hypersurface inside Riemannian manifolds. While Brendle's proof of Lawson conjecture didn't directly involve the flow however it did use some techniques coming from his sharp estimate analysis of the inscribed radius in noncollapsing. 

\section*{Organization}

The thesis is divided into three chapters where the first chapter serves as an introduction to the mean curvature flow. The main result in Chapter 1 is Huisken's theorem proving that no singularities appear for convex hypersurfaces evolving under mean curvature flow. Another crucial result proved is the monotonicity formula. 

Following this we study the asymptotic properties of the flow in the more general mean-convex setting in Chapter 2. Huisken-Sinestrari proved that asymptotically the flow converges to a weakly convex hypersurface.  

Chapter 3 is on Noncollapsing of mean-convex hypersurfaces. By a result of Sheng-Wang and Andrews, the noncollapsing is preserved under mean curvature flow. The proof goes through deriving a differential inequality for 

\section*{Acknowledgments}

I would like to thank my adviser Ben Andrews for introducing me to this beautiful subject. 

Special thanks to CMI walkblogstalklock club for the frequent walks around the campus and G-12 group for the Marina trips. 