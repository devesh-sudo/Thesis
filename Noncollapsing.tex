\chapter{Noncollapsing}

Noncollapsing in mean curvature flow is a powerful result which gives a geometric idea about the structure of singularities. It can be used to rule out certain singularity profiles for mean convex mean curvature flow. 

\section{Inscribed curvature}

Let $ \mathcal{M} \in \Rn $ be a smooth hypersurface which is the boundary of an open $ \Omega $. For $ x \in \mathcal{M} $, we want to find the radius of the largest inscribed sphere in $ \mathcal{M} $ touching it at $ x $. For any $ y \in \mathcal{M}\backslash \{x\} $, the radius of the sphere passing through $ x $ and $ y $ and touching $ \mathcal{M} $ at $ x $ is given by \begin{equation}
    r(x,y) = \frac{||x-y||^{2}}{2\left< x-y,N(x) \right>}
\end{equation}
where $ N(x) $ is the outward unit normal vector of $ \mathcal{M} $ at $ x $. To get the \textbf{inradius} which would be radius of the largest sphere inscribed in $ \mathcal{M} $, we take the infimum over all points $ y \in \mathcal{M}\backslash \{x\} $ to get \begin{equation}
    r(x) = \inf_{{y \in \mathcal{M}\backslash \{x\}}} r(x,y)
\end{equation}
Similarly the \textbf{inscribed curvature} $ k : \mathcal{M} \to [0, \infty) $  is given by the reciprocal of the inradius, \begin{equation}
    k(x) = \frac{1}{r(x)} = \sup_{y \in \mathcal{M}\backslash \{x\}} \frac{2\left< x-y,N(x) \right>}{||x-y||^{2}}
\end{equation}

\begin{defn}
    A mean convex hypersurface $ \mathcal{M} $ is said to be \textbf{$ \alpha $-noncollapsed}   if for every $ x \in \mathcal{M} $ there exists an open ball $ B $ of radius $ \frac{\alpha}{H(x)} $ entirely contained in $ \Omega $. In terms of inscribed curvature, this is same as the inequality \begin{equation}
        k(x) \ge \alpha H(x) \quad \text{ for all} \quad x \in \mathcal{M}.
    \end{equation}
\end{defn}


\begin{thm}[Noncollapsing]
    Let $ X: M^{n} \times [0,T) \to \Rn $ be a smooth solution of the mean curvature with $ X(\cdot, 0) = \mathcal{M}_{0} $ compact and $ \alpha $-noncollapsed. Then $ X(\cdot,t) = \mathcal{M}_{t} $ is $ \alpha $-noncollapsed for all $t \in [0,T)  $.  
\end{thm}

We'll prove a stronger result which will imply noncollapsing. 

\begin{thm}
    Let $ X : M^{n} \times [0,T) \to \Rn $ be a smooth solution of the mean curvature flow with $ \mathcal{M}_{0} $ properly embedded. Then \begin{equation}
        \frac{\partial}{ \partial t}k \le \Delta k -2 \sum_{\kappa_{i}<k}^{} \frac{(\nabla_{i}k)^{2}}{k- \kappa_{i}} 
    \end{equation}
    where in inequality holds in the viscosity sense. 
\end{thm}